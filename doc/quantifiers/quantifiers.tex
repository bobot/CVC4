\documentclass{llncs}
\usepackage{amsmath}
\usepackage{latexsym}
\usepackage{amssymb}
\usepackage{comment}
%\usepackage{fullpage}
\usepackage{proof}
\usepackage{url}
\usepackage[pdftex]{graphicx}
\usepackage[usenames]{color}
\usepackage{capt-of}
\usepackage{program}

\newtheorem{thm}{Theorem}
\newtheorem{cor}{Corollary}

\newcommand{\interp}[1]{[ \negthinspace [ #1 ] \negthinspace ]}

\newcommand{\ednote}[1]{{\bf [#1]}\message{ednote!}}

\newcommand{\bassert}{|assert|_\mathrm{base}}
\newcommand{\iassert}{|assert|_\mathrm{ind}}
\newcommand{\bentail}{|entailed|_\mathrm{base}}
\newcommand{\ientail}{|entailed|_\mathrm{ind}}
\newcommand{\breset}{|reset|_\mathrm{base}}
\newcommand{\ireset}{|reset|_\mathrm{ind}}
\newcommand{\bcex}{|cex|_\mathrm{base}}
\newcommand{\icex}{|cex|_\mathrm{ind}}
\def\LOOP{\qtab\keyword{loop}\ }
\def\ENDLOOP{\untab}
\def\REPEAT{\qtab\keyword{repeat}\ }
\def\UNTIL{\untab\keyword{until}\ }
\def\IF{\qtab\keyword{if}\ }
\def\THEN{\ \keyword{then}\ }
\def\ELSE{\untab\qtab\keyword{else}\ }
\def\ELSEIF{\untab\qtab\keyword{else if}\ }
\def\FI{\untab}
\def\RETURN{\keyword{return}\ }
\def\ENDPROC{\untab}
\def\DOFOR{\qtab\keyword{for}\ }
\def\ENDFOR{\untab}
\def\COMMENT#1{\texttt{// #1}}
\def\keyword#1{\mbox{\normalshape\bf #1}}
\def\MATCH{\qtab\keyword{match}\ }
\def\WITH{\ \keyword{with}\ }
\def\ENDMATCH{\untab}
%\def\WHILE{\qtab\keyword{while}\ }
\def\ENDWHILE{\untab}

\newcommand{\To}{\Rightarrow}


\begin{document}
\pagenumbering{arabic}
\pagestyle{plain}

\title{Counterexample-Based Quantifier Instantiation for SMT}

\author{Andrew Reynolds}
\institute{Computer Science, The University of Iowa, USA}

\date{}

\maketitle
\thispagestyle{empty}

\section{DPLL(T) Approach to Quantifiers}

When extending the DPPL(T) framework to handle formuls involving quantifiers, we rely on the following rules: \\

\noindent $\exists$-Inst: \\

$M \parallel F \Longrightarrow M \parallel F, \neg \exists \bar{x}. \varphi[ \bar{ x } ] \vee \varphi( \bar{ c } )$, if   
$\begin{cases}
  \exists \bar{x}. \varphi[ \bar{ x } ] \in M \\
  \bar{ c } \text{ are fresh constants} \\
\end{cases}$ \\

\noindent $\forall$-Inst: \\

$M \parallel F \Longrightarrow M \parallel F, \neg \forall \bar{x}. \varphi[ \bar{ x } ] \vee \varphi( \bar{ s } )$, if   
$\begin{cases}
  \forall \bar{x}. \varphi[ \bar{ x } ] \in M \\
  \bar{ s } \text{ are ground terms} \\
\end{cases}$ \\

We call the constants introduced in $\exists$-Inst as skolem constants.
Note that in the case of $\exists$-Inst, there is no benefit of instantiating more than once.

Note that the $\forall$-Inst rule always applies in DPLL(T) when a universal quantified formula is asserted, thus restricting the SMT solver from answering SAT.
Typically, repeated applications of $\forall$-Inst are tried until a conflict is discovered, or else the solver returns UNKNOWN.
The main challenges here are (1) determining relevant instantiations to use for the rule $\forall$-Inst, (2) determining when it is no longer worthwhile to apply $\forall$-Inst for a given quantifier, and furthermore (3) recognizing cases where \emph{all} necessary instantiations have been tried.

In current approaches, $E$-matching has been used as a method of addressing challenge (1).
In $E$-matching, subterms contained in a quantifier are matched against ground terms to find substitutions that are then used to instantiate a given quantifier.
While simple and effective, current approaches to $E$-matching are limited in their ability to determine the instantiations that are the most helpful at any given time.
As a result, often a brute force approach is used, in which many useless instantiations are produced in addition to those used to generate conflicts.

Approaches based on $E$-matching also suffer at times from what is known as a \emph{matching loop}, that is, a situation in which ground terms generated by instantiations may lead to matches of a repeating form.
Heuristics have been used to address the challenges of (2), including assigning an instantiation level to each ground term produced as a result of an instantiation.
By giving preference to matches using ground terms of a lower instantiation level, our instantiations effectively can be thought of as a bredth-first search.

Some SMT solvers employ what is known as \emph{complete instantiation} when dealing with problems containing quantifiers to address (3).
If the solver can recognize that the quantifier occurs in a decidable fragment of first order logic, then the solver may only need to try a finite number of instantiations before either finding a contradiction or determining the quantified formula is valid.
In the cases where the formula is found to be valid, then the solver may be able to answer SAT.

The following document reexamines challenges (1), (2) and (3).
We present a scheme for symbolically reasoning about counterexamples to quantified formulas within the DPLL(T) framework.
At a high level, for every (universally) quantified formula asserted in a given context, the solver will assume that a counterexample exists, and reason about possible values for this counterexample.

In terms of challenge (1), the goal of this approach will be to leverage theory-specific information to aid in selecting instantiations.
Say we have a universally quantified formula $\forall x. \varphi[x]$ where $x$ is of a sort belonging to theory $T$.
The SMT solver will be interested in finding a satisifying assignment for $ \neg \varphi[e]$ where $e$ is a fresh constant of the same sort as $x$.
In other words, the value of $e$ will represent the possible values for which our quantified formula can be falsified.
The theory solver for $T$ will suggest instantiations for $x$ based on the internal information it has deduced involving $e$.
For the theory of $EUF$, this can be thought of as an extension to E-matching.

For challenge (3), it can be shown that if $\neg \varphi[e]$ is unsatisfiable in the current context for fresh constants $\bar{e}$, then our quantified formula is valid in the current context and need not be instantiated further.
This comes as an added benefit, and while not as powerful as techniques using complete instantiation, may lead to SAT instances when it is shown that a particular axiom does not apply in a particular context.

\subsection{Related Work}

\paragraph{Model-Based Quantifier Instantiation}
Model-Based Quantifier Instantiation (MBQI) has been proposed as a powerful method for addressing challenges (1) and (3).
For a quantified formula $\forall \bar{x}. \varphi(\bar{x})$
The goal of MBQI is to determine relevant instantiations using models to ground clauses in the current context.
More specifically, given a set of clauses $F$, we determine a candidate model $M^I$ for the ground clauses in $F$, as well as other restrictions $R$ on potential models.
For each non-ground clause $C[x] \in F$, we check the satisifiability of $R \wedge \neg C^I[e]$, where $C^I[e]$ is generated by replacing all uninterpretted symbols in $C[x]$ with the interpretation in our model $M^I$.
In the case that this formula is satisfiable for some $C^I[e]$, we use a model for this formula to instantiate $C[x]$, thereby ruling out the model $M^I$ on future iterations.
In the case that each of these formulas $R \wedge \neg C^I[e]$ are unsatisifiable, then we know that our formula is SAT.

%However, a weakness of this approach is that it is not incremental.
%In other words, testing the satisfiability of formulas $R \wedge \neg C^I[e]$ potentially comes at a large cost to the solver.

The technique proposed in this document takes a similar approach to quantifier instantiation, and is intended to simplify this scheme.
Instead of explicitly building models $M^I$ and reasoning about the interpretted form of clauses $\neg C^I[e]$, we instead reason \emph{incrementally} about the negated body of non-ground clauses $\neg C[e]$ as they are, for a fresh (distinguished) constant $e$ \footnote{ We will refer to these constants $\bar{e}$ as \emph{instantiation constants} }.
This is done using the same instance of the SMT solver, that is, we may use an SMT solver to simultaneously reason about constructing satisfying assignments for ground clauses existing in a particular problem as well as clauses $\neg C[e]$.
In this approach, when we have reached a DPLL(T) state $M \parallel F$, our current set of asserted literals $M$ not containing $\bar{e}$ can be considered the implicit specification of a current model.

Based on possible values we find for $e$, our procedure will construct instantiations that are likely to be relevant in the current context, as well as decide when no helpful instantiations exist.
In the same manner as model-based quantifier instantiation, if we show that $\neg C[e]$ is unsatisifiable in the current context for the fresh constant $e$, then we have shown that $C[x]$ is valid and need not be instantiated further in the current context.

\paragraph{E-matching}
E-matching is a powerful method for generating relevant instantiations based on matching \emph{triggers} with concrete ground terms. 
Say we have ground assertions $L$ and quantified assertion $\forall \bar{x}. \varphi$.
The idea of E-matching is to find a subterm (or set of subterms) $t$ of $\varphi$, called a \emph{trigger}, such that $\bar{x} \in FV( t )$ and ground terms $s$ such that $t[\bar{s}/\bar{x}]$ is equivalent to a ground subterm $t'$ of $L$ modulo a set of ground equalities entailed by $L$.
In this case, we say that term $t$ E-matches term $t'$ and subsequently the set of ground terms $\bar{s}$ is used to instantiate the quantifier $\forall \bar{x}. \varphi$.

While modern methods can be used to compute E-matches in a highly efficient manner, a weakness of E-matching is its ability to recognize the relavance of instantiations.
In particular, a basic implementation of E-matching does not consider the logical structure of the body of the quantifier $\varphi$.
As a result, many instantiations may potentially be needed to generate a conflict if one exists.

For example, say we have the quantified formula $\forall x. ( f(x) \neq g(x) \Rightarrow P( x ) )$ and ground literal $f( a ) = g( a )$ asserted in a particular context. 
An implementation of E-matching may choose $f( x )$ as a trigger and will instantiate this formula with ground terms $a$ such $f( a )$ exists in $L$.
However, note in this case that this instantiation is useless since $f( a ) = g( a )$.

For this reason, it is useful to reason about the negated body of the quantifier for a fresh instantiation constant $e$, where in this case the SMT solver will assert the literals $f( e ) \neq g( e )$ and $\neg P( e )$ to find a satisifying assignment.
This will signify that we should be looking for a ground term $a$ such that $f( a ) \neq g( a )$, and ideally an $a$ in which $\neg P( a )$ holds as well.

The technique in the following can be thought of as an extension to E-matching, where now equalities and disequalities between triggers \footnote{We now consider a trigger as a ground term $t[e/x]$ containing a distinguished instantiation constant $e$. }
 are considered.
This information provides a criteria for filtering irrelevant E-matches.

\section{DPLL(T) with Counterexample Literals}
\label{dpll-ce-lit}

In this section, we propose a scheme for reasoning about quantified formulas in SMT within the existing DPLL(T) framework.
This scheme will supplement DPLL(T) with additional rules that provide ways for finding relevant instantiations.

\subsection{Counterexample Literals}

Given an asserted quantifier $\forall \bar{x}. \varphi[\bar{x}]$ (call this formula $\psi$) and a DPLL(T) context $M \parallel F$, we are interested in finding a satisifying assignment for $ \neg \psi[\bar{e}/\bar{x}]$ for fresh constants $\bar{e}$.
To do this, we will insert a valid lemma into $F$ that can be thought of saying that either (a) $\psi$ is not asserted, (b) $\psi$ does not have a counterexample, or (c) $\psi$ has a counterexample, call it $\bar{e}$ such that $\neg \varphi[\bar{e}]$ holds.
For this purpose, we will first introduce the notion of a \emph{counterexample literal}.

For a quantified formula $\psi$ of the form $\forall \bar{ x }. \varphi$, we will associate a distinguished (boolean) literal called the \emph{counterexample literal of $\psi$}, and write $\bot^{\neg \psi}$ to denote this literal.
This literal states that ``a counterexample to $\psi$ does not exist", or in other words $\psi$ is valid in the current context.
The idea here is that although this entails the literal $\psi$ itself, we will in a sense trick the SAT-solver into believing they are in fact independent literals.

For a DPLL(T) state $M \parallel F$ and for each quantified formula $\psi = \forall \bar{ x }. \varphi[ \bar{ x } ] \in F$, we have three intended configurations for $\psi$ and its counterexample literal $\bot^{\neg \psi}$: \\

(1) $\psi$ is not asserted positively in $M$, 

(2) $\psi^{(d)}$ and $\neg (\bot^{\neg \psi})^d$ are asserted in $M$, 

(3) $\psi^{(d)}$ and $\bot^{\neg \psi}$ are asserted in $M$. \\

In case (1), our configuration does not claim $\psi$ to be true.
In case (2), our configuration has asserted (perhaps as a decision) that $\psi$ is true, but also has decided that it will find a counterexample.
In case (3), our configuration has asserted that $\psi$ is true, and moreover knows that $\psi$ does not have a counterexample.

To enforce this schema, we require that the DPLL(T) engine never assert $(\bot^{\neg \psi})^d$ as a decision, and that $\neg (\bot^{\neg \psi})^d$ should be asserted only after $\psi$ is asserted.
Both these requirements can be addressed within the core of the SAT solver with only a limited impact on performance.
It also should be noted that neither of these are required to maintain soundness of the overall DPLL(T) system.

%In the following sections, lemmas of the form $( \neg \psi \vee \bot^{\neg \psi} \vee \neg \psi[\bar{e}/\bar{x}])$ will be added to our set of clauses $F$.
%Each of these clauses tell us that either $\psi$ does not hold, a counterexample does not exist for $\psi$, or the contraint $C(\bar{e})$ holds for the counterexample $\bar{e}$ of $\psi$,  where $\bar{e}$ are distinguished free constants of the same sorts as $\bar{x}$.
%In other words, if there exists a $\bar{t}$ such that $\psi[\bar{t}/\bar{x}]$ is unsatisifiable in $M$, then $C( \bar{t})$ is satisfiable in $M$.
%Note the contrapositive: if $C( \bar{e} )$ is unsatisfiable in $M$, then $\psi[\bar{t}/\bar{x}]$ is satisifiable in $M$ for all $\bar{t}$, or in other words, $\psi$ does not have a counterexample and $\psi$ is valid.

Consider the possible states (1), (2) and (3) of our configuration.
In case (1), if $\psi$ is asserted negatively, then the remainder of the clause $( \neg \psi \vee \bot^{\neg \psi} \vee \neg \psi[\bar{e}/\bar{x}])$ is ignored.
In the case of (3), the clause is satisfied by $\bot^{\neg \psi}$ and the solver will again ignore the information contained in $\neg \psi[\bar{e}/\bar{x}]$.

More interestingly, in the case of (2), the SAT-solver must find a satisfying assignment for $\neg \psi[\bar{e}/\bar{x}]$.
In particular, we will see that this will force the DPLL(T) engine to reason about models for our counterexample $\bar{e}$.
Furthermore, in the case where $\neg \psi[\bar{e}/\bar{x}]$ is unsatisifiable, this will signal for the SAT-solver to backjump and assert $\bot^{\neg \psi}$ positively as a non-decision literal, or in other words, a counterexample to $\psi$ cannot exist in the current context.
We will see that this will allow us to answer SAT instead of UNKNOWN in some cases.

\subsection{Instantiation Constants}

We will refer to distinguished free constants $\bar{e}$ as introduced in the previous section as \emph{instantiation constants}.
It is important to note that instantiation constants will have identical logical semantics as free constants of the same sort.
In other words, our scheme assumes no information about the value of an instantiation constant $e$, and simply uses this terminology as a way of referring such constants.

We write $e \mapsto^I_i \forall \bar{x}. \varphi[ \bar{x} ]$ to denote that $e$ is the i$^{th}$ instantiation constant for the formula $\forall \bar{x}. \varphi[ \bar{x} ] $, and write $\bar{e} \mapsto^I \forall \bar{x}. \varphi[ \bar{x} ]$ to denote that $e_1 \mapsto^I_1 \varphi \ldots e_n \mapsto^I_n \forall \bar{x}. \varphi[ \bar{x} ]$.
For a literal $l$, we write $l \mapsto^I S$ to denote that $S = \{ \forall \bar{x}. \varphi[ \bar{x} ] \mid \exists e \in l. e \mapsto^I \forall \bar{x}. \varphi[ \bar{x} ] \}$, or in other words, all instantiation constants in $l$ refer to a quantified formula in $S$.
Furthermore write $C \mapsto^I S$ to denote that $S$ is the union of the sets $S_1 \ldots S_n$ for $l_i \in C$, $l_i \mapsto^I S_i$.
A concrete literal (clause) is one that contains no instantiation constants.
We consider a literal (clause) to be pure with respect to quantifiers if it contains instantiation constants from at most one quantified formula.
For convienience, we simply write $E \mapsto^I \forall \bar{x}. \varphi[ \bar{x} ]$ in the case that expression $E$ contains instantiation constants from exactly one.
An expression $E$ is \emph{concrete} if $E \mapsto^I \emptyset$.

\subsection{Counterexample Lemma for Quantified Formulas}

Consider the lemma $( \neg \psi \vee \bot^{\neg \psi} \vee \neg \psi[\bar{e}/\bar{x}])$, where $\bar{e} \mapsto^I \psi$.
We will refer to this formula as the \emph{counterexample lemma for $\psi$}.
Informally, this lemma is valid since $\bot^{\neg \psi}$ entails $\psi$, and thus $( \neg \psi \vee \bot^{\neg \psi} )$ is a tautology.
Note that $\neg \psi[\bar{e}/\bar{x}]$ may contain quantifiers.

{\bf Example 0}
Consider the formula $\psi = \forall x. ( l_1 \vee ( l_2 \vee x = a ) \Rightarrow \neg P( x ) )$.
Its counterexample lemma after conversion to CNF is $( \neg \psi \vee \bot^{\neg \psi} \vee \neg l_1 ) \wedge ( \neg \psi \vee \bot^{\neg \psi} \vee l_2 \vee e = a ) \wedge ( \neg \psi \vee \bot^{\neg \psi} \vee \neg P( e ) ) $. \\

In Example 0, we have produced three clauses.
The first says that either $l_1$ is false or a counterexample does not exist to $\psi$.
The second says that either $l_2$ is true, a counterexample $\psi[e/x]$ exists such that $e = a$, or a counterexample does not exist to $\psi$. 
Similarly, the third says that either a counterexample $\psi[e/x]$ exists such that $P( e )$, or one does not exist.

\subsection{Counterexample $\forall$-Inst}

The transformation described in Section~\ref{sec:CENF} can be used to introduce clauses containing instantiation constants into the DPLL(T) framework.
Conceptually, the following rule constructs a lemma that describes a counterexample $\bar{e}$ for the universally quantified formula $\forall \bar{x}. \varphi[ \bar{ x } ]$, and adds it as a lemma to $F$.
This rule represents the central idea behind our approach: \\

\noindent Counterexample $\forall$-Inst: \\

$M \parallel F \Longrightarrow M \parallel F, \neg \forall \bar{x}. \varphi[ \bar{x} ] \vee \bot^{\neg \forall \bar{x}. \varphi[ \bar{x} ]} \vee \neg \varphi[ \bar{ e } ]$, if   
$\begin{cases}
  \forall \bar{x}. \varphi[ \bar{ x } ] \in M \\
  \bar{ e } \mapsto^I \forall \bar{x}. \varphi[ \bar{ x } ] \\
\end{cases}$ \\

Let us again consider Example 0.
By applying Counterexample $\forall$-Inst to $\psi$, we will produce three clauses $C_1, C_2, C_3$ after CNF-conversion:

$C_1 : ( \neg \psi \vee \bot^{\neg \psi} \vee \neg l_1 )$,

$C_2 : ( \neg \psi \vee \bot^{\neg \psi} \vee l_2 \vee e = a )$, and

$C_3 : ( \neg \psi \vee \bot^{\neg \psi} \vee \neg P( e ) )$.

If we force the SMT solver to assert $\neg( \bot^{\neg \varphi} )^d$, it can be shown that the default behavior of DPLL(T) is sufficient and desirable for reasoning about clauses containing counterexample literals and literals with instantiation constants.

Let us first consider when we are in a state $M, \psi, \neg( \bot^{\neg \varphi} )^d, N \parallel F$, were $\psi$ has been asserted as a non-decision literal and its counterexample literal has been decided upon negatively.

In the case that $l_1$ is asserted positively (possibly as a decision) in $M = M_1, l^{(d)}_1, M_2$, we have that $C_1$ is falsified and we will backjump to the state $M, \psi, \bot^{\neg \varphi} \parallel F$.
This corresponds to saying that a counterexample cannot exist to $\psi$ in the context $M$ where $l_1$ is true.

In the case that $l_1$ is not asserted in $M$, we will apply Unit Propagation to $C_1$ and arrive in the state $M, \psi, \neg( \bot^{\neg \varphi} )^d, \neg l_1$ $\parallel F$.
This corresponds to saying that $l_1$ must be false if a counterexample to $\psi$ exists.
Note here that we are preferring to constrain ourselves to states in which a counterexample may occur.
This makes sense, both semantically since we have decided that a counterexample may exist (i.e. $\neg( \bot^{\neg \varphi} )^d$), and operationally since the DPLL(T) will reach a non-terminating state if it cannot prove that a counterexample exists.

Otherwise, if $l_1^d$ had been asserted positively as a decision in $N$, $C_1$ is falsified and conflict analysis would proceed in the standard way to backjump to a state of the form $M, \psi, \neg( \bot^{\neg \varphi} )^d, N', \neg l_1 \parallel F$ or $M, \psi, \neg( \bot^{\neg \varphi} )  \parallel F$.
In the former case, we have determined that some decision in $N$ has led us to require $l_1$ to be true thereby conflicting with our decision that a counterexample exists.
In the latter case, we have determined that assuming a counterexample exists means that $l_1$ must be true and thus a counterexample cannot exist.

%Now consider the case where we are in a state $M, \psi^d, \neg( \bot^{\neg \varphi} )^d, N  \parallel F$ where now $\psi$ has been asserted as a decision literal.

%The cases here are identical with one notable exception.
%Take the case where we have encountered a conflict and backjumped to the state $M, \psi^d, \bot^{\neg \varphi} \parallel F$, that is, we know that $\psi$ does not have a counterexample.
%Note the lemma $(\neg (\bot^{\neg \psi}) \vee \psi)$, that is, either $\psi$ has a counterexample or $\psi$ holds.
%We can apply Backjump with $C' \vee l' = \neg (\bot^{\neg \psi}) \vee \psi$ to arrive at the state $M, \psi \parallel F$.
%Note, however that this lemma should not be added $F$, as it is exactly what we \emph{do not} want to tell the SAT-solver, as it encodes the trick of our approach.

The analysis for $C_2$ and $C_3$ are as expected, where now the DPLL(T) engine may give partial truth assignments to $l_2$, $e = a$ and $P( e )$ which will be useful for determining instantiations.
Note in the context $M, \psi, \neg( \bot^{\neg \varphi} )^d, \neg l_2$, the literal $e = a$ will be unit propagated within $C_3$, thereby signalling that $\psi[ a/x]$ is the \emph{only} instantiation we need to try in this example.

We formally define the following invariants for DPLL(T): \\

{\bf Invariant 1}:
For all DPLL(T) states $M \parallel F$, the only instance of literal $\bot^{\neg \psi}$ in $F$ occur in clauses $C_1 \ldots C_n$ such that $C_1 \ldots C_n$ is logically equivalent to $( \neg \psi \vee \bot^{\neg \psi} \vee \neg \psi[\bar{e}/\bar{x}] )$ where $\bar{e} \mapsto^I \bar{x}$.

{\bf Invariant 2}:
For all DPLL(T) states $M \parallel F$, all clauses $C \mapsto^I \psi$ in $F$ are tautological. \\

By definition, counterexample $\forall$-Inst maintains Invariants 1 and 2.
In addition, since counterexample literals $\bot^{ \neg \psi }$ have no meaning apart from representing boolean values, no lemma added to $F$ by $T$-Learn will violate Invariant 1.

\subsection{Enhancements for DPLL(T) Rules}

First, let us revisit the rule for $\forall$-Inst.
Note that we may restrict this rule to only be applicable to formulas $\forall \bar{x}. \varphi[ \bar{ x } ]$ in cases where a counterexample can exist to $\forall \bar{x}. \varphi[ \bar{ x } ]$.
Such cases can be easily recognized due to the following theorem:

\begin{thm}
For all DPLL(T) states of the form $M, \bot^{\neg \psi}, M' \parallel F$, if no literal $l \mapsto^I \psi$ is asserted as a decision in $M$, then $M, \bot^{\neg \psi}, M' \models_T \psi[\bar{t}/\bar{x}]$ for all $\bar{t}$.
\end{thm}
\begin{proof}
Assume we have a DPLL(T) state $M, \bot^{\neg \psi}, M' \parallel F$, where $\bot^{\neg \psi}$ is asserted as a non-decision literal.
Say that $\psi$ has DNF-conversion $\forall \bar{x}. (S_1 \vee \ldots \vee S_n)$, and thus $F \models ( \neg \psi \vee \bot^{\neg \psi} \vee (\neg S_i)[\bar{e}/\bar{x}])$ for all $1 \leq i \leq n$.
By Invariant 1, the literal $\bot^{\neg \psi}$ only occurs in such clauses in $F$.

Consider the case where $\bot^{\neg \psi}$ was asserted by unit-propagation.
Then $M \models \neg ( \neg \psi \vee (\neg S_i)[\bar{e}/\bar{x}])$ for some $S_i$.
Therefore, we know $\psi \in M$, and since no literal $l \mapsto^I \psi$ is asserted as a decision in $M$, 
we can construct an explanation $M'' \subseteq M$ such that $M''$ does not contain $\bar{e}$, and $M'' \models_T \neg (\neg S_i)[\bar{e}/\bar{x}]$.
Therefore, we know that $M'' \models_T \neg \exists \bar{x}. (\neg S_i)$ since $\bar{e} \not\in M''$, and thus $M'' \models_T \forall \bar{x}. S_i$.
Since $M'' \subseteq M$, we have that $M, \bot^{\neg \psi}, M' \models_T \psi[\bar{t}/\bar{x}]$ for all $\bar{t}$.

Consider the case where $\bot^{\neg \psi}$ was asserted due to a backjump.
[do this] $\Box$
\end{proof}

Accordingly, we add the following restriction to our DPLL(T) engine: \emph{if $(\bot^{\neg \psi})^d$ is not asserted, then no literal $l^d$ can be asserted such that $l \mapsto^I \psi$}.

Our updated rule for $\forall$-Inst becomes the following: \\

\noindent $\forall$-Inst (Counterexample Restricted): \\

$M \parallel F \Longrightarrow M \parallel F, \neg \forall \bar{x}. \varphi[ \bar{ x } ] \vee \varphi( \bar{ s } )$, if   
$\begin{cases}
  \forall \bar{x}. \varphi[ \bar{ x } ] \in M \\
  \bar{ s } \text{ are \emph{concrete} ground terms} \\
  \bot^{\neg \forall \bar{x}. \varphi[ \bar{ x } ]} \text{ is not asserted positively }\\
  \text{ \ \ \ as a non-decision literal in $M$ } \\    
\end{cases}$ \\

Note that these restrictions are helpful when trying to establish a satisifiable instance of a formula involving quantifiers.
The following example shows an instance where the solver may return SAT by determining that a counterexample cannot exist to a universally quantified formula in a particular context. \\

{\bf Example 1}
Say we wish to determine the satisifiability of the set of formulas $S = \{ a = b, (\psi :) \forall x. ((f(x) = a \wedge f(x) \neq b) \Rightarrow l) \}$.
After two applications of Unit Propagation, we arrive in the DPLL(T) state $( a = b ), \psi \parallel S$.
After using Counterexample $\forall$-Inst, our state becomes $( a = b ), \psi \parallel (S':) S \cup \{ (\neg \psi \vee \bot^{\neg \psi} \vee f( e ) = a), (\neg \psi \vee \bot^{\neg \psi} \vee f( e ) \neq b)), (\neg \psi \vee \bot^{\neg \psi} \vee \neg l) \}$.
Say the solver decides that $\psi$ has a counterexample, and our state becomes $( a = b ), \psi, \neg (\bot^{\neg \psi})^d \parallel S'$. 
After unit propagation our state becomes $\ldots, f( e ) = a, f( e ) \neq b, \neg l \parallel S'$, and we encounter the EUF theory conflict $a = b, f( e ) = a, f( e ) \neq b \models \bot$.
After conflict analysis, we will backtrack to the state $( a = b ), \psi, (\bot^{\neg \psi}) \parallel S'$, and now notice that all clauses in $S'$ are satisfied and $\forall$-Inst does not apply to $\psi$.
The solver answers SAT. \\

Here, we have determined $S$ is satisfiable by noting that any model where $a = b$ is true also satisfies $\psi$ for all possible values for $x$.


\subsection{Representation of Counterexamples}

It is important to note that in the DPLL(T) search we branch on literals involving instantiation constants.
Thus, at any given point we are searching for only one possibility of how a counterexample to a quantified formula $\psi$ can look.
In this section, we describe precisely what is said about our counterexample to $\psi$.

We say a DPLL(T) state $M \parallel F$ is \emph{candidate-satisifiable} if all clauses in $F$ are satisified and $M$ is $T$-consistent.
We say a quantified formula $\psi \in M$ is \emph{counterexample-active} in $M \parallel F$ if Counterexample $\forall$-Inst has been applied to $\psi$ and $\neg (\bot^{\neg \psi})^d \in M$.
The following theorem tells us what formulas are asserted for counterexample-active $\psi$ in candidate-satisifiable DPLL(T) states:

\begin{thm}
\label{thm:cerep}
Given a candidate-satisifiable DPLL(T) state $M \parallel F$ and a quantified formula $\psi \in M$ whose CNF-conversion is $\forall \bar{x}. (C_1 \wedge \ldots \wedge C_n)$.
If $\psi$ is counterexample-active in $M \parallel F$ for $\bar{e} \mapsto^I \psi$, then there exists a clause $C_i$ such that each literal in $C_i[ \bar{e}/\bar{x}]$ is asserted falsely in $M$.
\end{thm}

\subsection{Learned Lemmas Containing Instantiation Constants}

It is also important to note that learned lemmas can involve instantiation constants.
In Example 1, after the theory conlifct, we could have added the lemma $(\varphi_1 : ) ( f( e ) \neq a \vee f( e ) = b \vee a \neq b )$ to $S'$.
For a larger example, this lemma may potentially be useful in pruning the search space of subsequent searches.

\begin{comment}
It is also important to note that learned lemmas can involve instantiation constants.
In this example, after the theory conlifct, we could have added the lemma $(\varphi_1 : ) ( f( e ) \neq a \vee f( e ) = b \vee a \neq b )$ to $S'$.
In this case, we may apply Decide to come to the state $( a = b ), \psi, (\bot^{\neg \psi}), (f( e ) \neq a)^d \parallel S' \cup \varphi$, where again all clauses are satisfied, and the solver answers SAT in the same manner.

However, note that the lemma $\varphi_1$ is only useful in contexts in which $\psi$ is asserted.
In other words, the solver should not be searching for values of counterexamples to quantified formulas $\psi$ when $\psi$ is not asserted.
We will see in Section~\ref{sec:implementation} a recommended implementation for which this concern is addressed.
[do this: do these lemmas pollute the DPLL(T) space?  When should we forget them?  Should we give them to minisat?]
[do this: examine elaborating lemmas involving instantiation constants as being lemmas involving universal statements, justify as well]
\end{comment}

\section{Quantifier Instantiation Using Instantiation Constants}

In this section, we describe search methods for determining relevant instantiations using theory-specific information we have deduced about instantiation constants $\bar{e}$.

We begin with the following definitions:
An instantiation constant $e$ is \emph{solved in $M$} if there exists a concrete ground term $s$ such that $M \models_T (e = s)$.
An instantiation constant $e$ is \emph{unsolved in $M$} if $e$ is in $M$ and there does not exist a concrete ground term $s$ such that $M \models_T (e = s)$.
Similarly, a term $t$ is \emph{unsolved in $M$} if it contains instantiation constants that are unsolved in $M$; a term $t$ is \emph{solved in $M$} otherwise.
Note that concrete ground terms $t$ are solved for all $M$.

Given a quantified formula $\psi$ is \emph{instantiation-ready in $M$} if and only if $M$ is $T$-consistent and all $e_i \mapsto^I \psi$ in $M$ are solved in $M$.
For an instantiation-ready quantified formula $\psi$, define $\varepsilon_M( e_1 ) \ldots \varepsilon_M( e_n )$ as a vector of concrete ground terms such that if $e_i$ is in $M$, then $M \models_T (e_i = \varepsilon_M( e_1 ))$.

The usefulness of determining whether a quantified formula $(\psi:) \ \forall \bar{x}. \varphi[ \bar{x} ]$ is instantiation-ready is immediate:  if $\psi$ is counterexample active in candidate satsifiable $M \parallel F$, we know that $\neg \varphi( \bar{e} )$ is entailed in our current model.
If we also know that $\bar{e} = \bar{s}$, then we know that the instantiation $\varphi[ \bar{s} ]$ must be \emph{unsatisfiable} in the current state.

The following theorem describes cases when we can construct a useful instantiation:

\begin{thm}
\label{thm:instready}
Given a candidate-satisifiable DPLL(T) state $M \parallel F$, for all quantified formula $\psi \in M$, if $\psi$ is counterexample-active in $M \parallel F$ and instantiation-ready in $M$, then $\psi[\varepsilon_{M'}( \bar{e} )/\bar{x}]$ is $T$-unsatisfiable in $M$, where $\bar{e} \mapsto^I \psi$.
\end{thm}
\begin{proof}
Consider the quantified formula $\psi \in M$ whose CNF conversion is $\forall \bar{x}. C_1 \wedge \ldots \wedge C_n$.
Assume $\psi$ is counterexample-active in $M \parallel F$ and instantiation-ready in $M$.
Note by Theorem~\ref{thm:cerep}, literals $\neg l_1 [\bar{e}/\bar{x}] \ldots \neg l_n [ \bar{e}/\bar{x} ]$ are asserted in $M$ for some $C_i = ( l_1 \wedge \ldots l_n )$

Using proof by contradiction, let us assume that $\psi[\varepsilon_{M}( \bar{e} )/\bar{x}]$ is $T$-satisfiable in $M'$.
Thus, at least one literal $l_j[ \varepsilon_{M}( \bar{e} )/\bar{x}] \in C_i$ must be $T$-satisfiable in $M$.
However, we have that $\neg l_j [\bar{e}/\bar{x}] \in M$ and since $\varepsilon_{M}( e ) = e$ for all $e \in M$, we know that $\neg l_j [\varepsilon_{M}( \bar{e} )/\bar{x}]$ is entailed by $M$.
Thus, $l_j[ \varepsilon_{M}( \bar{e} )/\bar{x}]$ is $T$-satisfiable in $M$, and $\psi[\varepsilon_{M}( \bar{e} )/\bar{x}]$ is $T$-unsatisfiable in $M'$. $\Box$
\end{proof}

\begin{cor}
\label{cor:instready}
Given a candidate-satisifiable DPLL(T) state $M \parallel F$, for all quantified formula $\psi \in M$,
if $\psi$ is counterexample-active in $M \parallel F$ and instantiation-ready in $M$, then the instantiation clause $( \neg \psi$ $\vee$ $\psi[\varepsilon_{M}( \bar{e} )/\bar{x}])$ has not been added to $F$.
\end{cor}
\begin{proof}
The proof is immediate, noting that by Theorem~\ref{thm:instready}, $( \neg \psi \vee \psi[\varepsilon_{M}( \bar{e} )/\bar{x}])$ is $T$-unsatisifiable in $M$, and therefore cannot be satisfied in $M \parallel F$. $\Box$
\end{proof}

In other words, Theorem~\ref{thm:instready} states that if we can find a state $M$ in which we have solved for values of all instantiation constants for $\psi$, we can construct an instantiation for $\psi$ that leads to a conflict in $M$.
Corollary~\ref{cor:instready} states that if we can find such an instantiation, then it has not yet been applied.
This fact guarentees that no redundant instantiations are used in our instantiation scheme.

\subsection{Quantifier Instantiation for EUF}

A method for quantifier instantiation in EUF is presented here.
This method uses information that is asserted as a result of searching for satisifying assignments to the negated bodies of quantifiers.
Given a quantified formula $\psi$ of the form $\forall x. \varphi[x]$ where $x$ is of an uninterpretted sort, searching for a satisfying assignment to $\neg \varphi[e]$, where $e$ is an instantiation constant, will provide a basis for determining what values should be used to instantiate $\psi$.

The EUF theory solver will process equalities and disequalities both between terms containing instantiation constants and those without.
Since instantiation constants have identical logical semantics to non-instantiation constants, equalities between terms containing instantiation constants may be treated identically, that is, the EUF theory solver is used as is.

Say the equality $t[e] = t'$ is entailed in the current context.
Here, we are interested in finding a ground term $r$ such that $M \models t[r] = s$ has maximum likelihood to hold.
If $M \models t[s] = t'$, then $s$ is a strong candidate for instantiation, if $M \models t[s] \neq t'$, then $s$ is useless as an instantiation for $x$, and otherwise if $t[s] = t'$ is satisifiable in $M$, then $s$ is potentially useful as an instantiation.
Similar to E-matching, we may be interested in determining if a term $t[q]$ exists as a subterm in another equality, regardless of whether it is equal to $t'$.

In relation to E-matching, a term containing instantiation constants can be thought of as a trigger for instantiation.
In contrast, however, we also are able to consider entailed equalities and disequalities in this approach.
This enables a powerful search method for determining relavant instantiations, as well as a criteria for judging the relevance of determined instantiations.
For the former, we can imagine an approach in which we first search only the equivalence class of $t[e]$ for another term of the form $t[s]$, and if none can be found, search the remainder of the concrete ground terms existing in the current context.
This is the idea of the approach proposed in the following section.
In addition, for the latter, we may filter instantiations $s$ if, for example, $s = e$ is unsatisifiable in the current context.

Our approach for quantifier instantiation for EUF will be divided into multiple iterations, where on each subsequent iteration, less restrictions are imposed upon the terms that are matched.
On the final iteration, our approach will be roughly equivalent to E-matching, where all triggers $f( \bar{t} )[\bar{e}]$ are matched against all concrete ground terms $g$ such that $M \not\models f( \bar{t} )[\bar{e}] \neq g$.

We begin our discussion by introducing necesary terminology.
We first describe the notion of a literal being induced by a set of equalities $E$.

Say we have an equality $f( \bar{t} )[\bar{e}] = g$ in $M$.
Our goal will be to find a set of equalities $\bar{e} = \bar{s}$ that are sufficient to show that $t[\bar{e}] = g$ must be true.
In other words, we want to find $\bar{s}$ such that $M \backslash f( \bar{t} )[\bar{e}] = g, \bar{e} = \bar{s} \models f( \bar{t} )[\bar{e}] = g$.
Note the subtle difference between this and E-matching.
In E-matching, we are searching for a substitution $\bar{s}$ such that $f( \bar{t} )[\bar{s}]$ is equivalent to $g$ modulo our set of ground equalities.
In this approach, we need only consider the equalities necessary to induce $f( \bar{t} )[\bar{e}]$ to be equal to $g$.

Consider the following simple example.
Say we have the quantifier $\psi$ of the form $\forall x. (g(x) = a \Rightarrow f( x, g( x ) ) \neq f( b, a ))$.
After applying Counterexample $\forall$-Inst, we will have the literals $g(e) = a$ and $f( e, g( e ) ) = f( b, a )$ asserted in the current context for $e \mapsto^I \psi$.
Note that now $e = b$ suffices to show that $f( e, g( e ) ) = f( b, a )$, thus suggesting that we may wish to use $b$ to instantiate $\psi$.
In a basic implementation of E-matching, $f( x, g( x ) )$ and $f( b, a )$ do not E-match.

More generally, we will say that a literal $t \sim s$ is $E$-induced in $M$ if $M \backslash t \sim s, E \models t \sim s$, where $E$ is a set of equalities of the form $e_1 = s_1 \ldots e_n = s_n$, that is, equalities whose left hand sides are instantiation constants.

To calculate if the equality $f( \bar{t} )[\bar{e}] = g$ is induced by a set of equalities $\bar{s} = \bar{e}$, we use the following method $calc\_e\_ind$:

\begin{minipage}[t]{.4\linewidth}
\begin{program}
\PROC |calc\_e\_ind|(t,g,M) \BODY
  \IF t \not\mapsto^I \psi \text{ or } M \models t \neq g \THEN
    \RETURN \emptyset;
  \ELSEIF t \text{ is inst. constant } \THEN
    \IF M \models t = g \THEN
      \RETURN \{ \emptyset \};
    \ELSE
      \RETURN \{ t = rep(g) \};
    \FI
  \ELSEIF t, g \text{ are $f( \bar{t_a} )$, $f( \bar{g_a} )$ } \THEN
    S := \{ \emptyset \};
    \DOFOR \text{each } t_i \text{ in } \bar{t_a} 
      \IF M \models t \neq g \THEN
        \RETURN \emptyset;
      \ELSEIF M \not\models t_i = g_i
        S' := calc\_e\_ind\_mod( t_i, g_i );
        S := merge( S, S' );
      \FI
    \ENDFOR
    \RETURN S;
  \ELSE
    \RETURN \emptyset;
  \FI
\ENDPROC
\end{program}
\end{minipage}
\begin{minipage}[t]{.4\linewidth}
\begin{program}
\PROC |calc\_e\_ind\_mod|(t,g,M) \BODY
  E := \emptyset;
  \DOFOR \text{each pair } t_i, g_i = t, g
    E := E \cup calc\_e\_ind( t_i, g_i, M );
  \ENDFOR
  \RETURN E;
\ENDPROC
\end{program}
\begin{program}
\PROC |merge|(S,S') \BODY
  S'' := \emptyset;
  \DOFOR E \in S, E' \in S'
    \IF E \text{ and } E' \text{ are compatible }
      S'' := S'' \cup \{ E \cup E' \}
    \FI
  \ENDFOR
  \RETURN S'';
\ENDPROC
\end{program}
\end{minipage}

The method $calc\_e\_ind(t,g,M)$ returns a set of equality sets each that induce $t = g$ in $M$.
In the case that $t$ is ground or disequal from $g$, the method returns the empty set.
In the case that $t$ is an instantiation constant, the method returns the set containg one set of equalities, containing $t = g$ if necessary.
If $t$ is of the form $f(\bar{t})$ and g is of the form $f(\bar{g})$, then the method computes the set $S$ such that $t = g$ is $E$-induced in $M$ for each $E \in S$.

In this case, the method scans each pair of arguments $t_i, g_i$ and computes, modulo equality, a set of equality sets $S'$ such that $t_i = g_i$ is $E$-induced in $M$ for each $E \in S'$, using the method $calc\_e\_ind\_mod$.
For each set $S'$ produced in this way, it merges this set with $S$ using the method $merge$.
This method pairwise combines compatible equality sets $E$ and $E'$ together, that is, it takes the union of equalities in $E$ and $E'$ where there does not exist an equality $e = t$ in $E$ and $e = t'$ in $E'$ and $t$ is not syntactically equivalent to $t'$.

Otherwise, if $t$ is of the form $f(\bar{t})$ and $g$ is not of the form $f( \bar{g})$, the method $calc\_e\_ind(t,g,M)$ returns the empty set.

We will also be interested in classifying which pairs of terms may potentially be $E$-induced.
For a quantified formula $\psi$, a term $f( t_1, \ldots, t_n ) \mapsto^I \psi$ is \emph{equality-compatible with term $s$ in $M$} if and only if $s$ is of the form $f( s_1, \ldots s_n )$ and $M \models_T t_1 = s_1 \wedge \ldots \wedge t_n = s_n$.
A term $f( t_1, \ldots, t_n ) \mapsto^I \psi$ is \emph{equality-independent from term $s$ in $M$} if (a) $s$ is of the form $g( s_1, \ldots s_m )$, or (b) $s$ is of the form $f( s_1, \ldots s_n )$ in $M$, and $M \models_T t_i \neq s_i$ for some $i$. 
A pair of terms $t, s$ where $t \mapsto^I \psi$ is \emph{equality-ambiguous} if $t$ is neither equality-compatible nor equality-independent from $s$.

\subsubsection{E-Matching Using Equalities and Disequalities Between Triggers}

We now describe the iterations used when determining instantiations for a quantified formula $\psi$ of the form $\forall \bar{x}. \varphi[\bar{x}]$, where $\bar{x}$ are of uninterpretted sorts.
We are interested in finding an instantiation $\bar{s}$ for $\psi$ in a candidate-satisfiable DPLL(T) state $M \parallel F$, where $\psi$ is counterexample-active in $M$ for $\bar{e} \mapsto^I \psi$.
In this case, a set of literals $L_1[\bar{e}] \ldots L_n[\bar{e}]$ have been asserted in $M$ according to Theorem~\ref{thm:cerep}.

Our strategy for instantiation will consist of four iterations.
On the first iteration, we return the instantiation $\varepsilon_M( \bar{e} )$ if and only if $\psi$ is instantiation-ready.
On the second iteration, we determine if there exists a set of terms $\bar{s}$ such that we can determine that at least one of $L_1[\bar{e}] \ldots L_n[\bar{e}]$ is $(\bar{s} = \bar{e})$-induced in $M$.
On the third iteration, we will search for equality-ambiguous pairs of terms $t[\bar{e}]$ and $s$ in the same equivalence classes.  
Specifically, we will construct an instantiation $\bar{s}$ such that an equality between a subterm position of $t[\bar{e}]$ and $s$ is induced by $\bar{s} = \bar{e}$ on this step.
On the fourth iteration, we find a set of terms $\bar{s}$ that induce at least one equality between a pair of terms $t[\bar{e}]$ and $g$.
Note that this step is roughly equivalent to basic E-matching.

\paragraph{First Iteration: Instantiation-Ready}

On the first iteration of quantifier instantiation for $\psi$, we check to see if each $e_i$ in $\bar{e}$ that is present in $M$ exists in the same equivalence class as a concrete ground term $\varepsilon_M(e_i)$.
For each $e_i$ not present in $M$, we pick an arbitrary concrete ground term if one exists, or a fresh constant otherwise.
If we can construct a vector of terms of this form, we return only this instantiation.
By Theorem~\ref{thm:instready}, this instantiation is \emph{guarenteed} to lead to a conflict, and the DPLL(T) engine will backtrack.

\paragraph{Second Iteration: A Literal is E-Induced in M}

If $\psi$ is not instantiation ready, we determine if any literal can be induced by a set of equalities $\bar{e} = \bar{s}$.
Here, we describe how to generate a set of equality sets $S_L$ for each literal $L[\bar{e}]$ such that $L[\bar{e}]$ is $E$-induced in $M$ for each $E \in S_L$.

Consider when $L[\bar{e}]$ is an equality of the form $t[\bar{e}] = s$, where $s$ is a concrete ground term.
In this case, we take $S_L$ to be the union of $calc\_e\_ind( t[\bar{e}], g, M )$ for each term $g$ in the equivalence class of $s$.

Otherwise consider when $L[\bar{e}]$ is an equality of the form $t[\bar{e}] = s[\bar{e}]$.
In this case, we scan each equivalence class and determine if there exists a pair of terms $g$ and $h$ in that equivalence class such that $t[\bar{e}] = g$ and $s[\bar{e}] = h$ are induced by sets of equality sets $S$ and $S'$.
If this is the case, we add $merge( S, S')$ to $S_L$.

Consider when $L[\bar{e}]$ is a disequality of the form $t[\bar{e}] \neq s$, the literal is converted to an equality between $Eq( t[\bar{e}], s )$ and $\bot$, where $Eq$ is a binary predicate representing equality between terms of boolean type, and $\bot$ is the distinguished boolean term representing false.
To determine $S_L$, we take $S$ to be the union of $calc\_e\_ind( Eq( t[\bar{e}], s ), h, M )$ merged with $calc\_e\_ind( Eq( s, t[\bar{e}] ), h, M )$ for each $h = \bot$.

This can be summarized in the following method:

\begin{program}
\PROC |calc\_e\_ind\_lit|(L,M) \BODY
  S := \emptyset;
  \IF L \text{ is } t[\bar{e}] = s \THEN
    \IF s \text{ is concrete ground } \THEN
      \DOFOR \text{each $g$, where $g = s$ } 
        S := S \cup calc\_e\_ind( t[\bar{e}], g );
      \ENDFOR
    \ELSEIF s \text{ is ground }
      \DOFOR \text{each } g, h \text{ where } h = g
        S := S \cup merge( calc\_e\_ind( t[\bar{e}], g ), calc\_e\_ind( s, h ) );
      \ENDFOR 
    \FI
  \ELSEIF L \text{ is } Eq( t[\bar{e}], s ) = \bot \THEN
    \DOFOR \text{each } t', s' \text{ where } Eq( t', s' ) \text{ or } Eq( s', t' ) = \bot
      S := S \cup merge( calc\_e\_ind( t[\bar{e}], t' ), calc\_e\_ind( s, s' ) );
    \ENDFOR
  \FI
  \RETURN S;
\ENDPROC
\end{program}

Note that the method $calc\_e\_ind\_lit$ generates \emph{partial} sets of equalities, or in other words, equality sets $E$ where $e_i = t$ need not be in $E$ for each $e_i$ in $\bar{e}$.
To construct a full instantiation $\bar{e} = \bar{s}$, we potentially need to add equalities for each instantiation constant $e_i$ not existing in $E$.

Thus, after the set of equality sets $S_L$ is computed for each $L[\bar{e}]$, we wish to determine if any instantiations can be constructed from these sets.
We perform a step in which equality sets are combined to construct sets that contain all instantiation constants $\bar{e} \mapsto^I \psi$, so that an instantiation can be constructed.
In terms of E-matching, this serves a similar purpose as \emph{multi-triggers}.

We call an equality set $E$ \emph{complete for $\psi$} if it contains an equality $e_i = t_i$ for each unsolved $e_i \mapsto^I \psi$, \emph{incomplete for $\psi$} otherwise.
For a complete equality set $E$, we define the equality set $E^\varepsilon$ as the equality set containing $E$ as well as equalities $ e = \varepsilon_M( e )$ for each solved $e \mapsto^I \psi$ not existing in $E$.

Overall, the second iteration computes $calc\_e\_ind\_lit$ for each literal $L[\bar{e}]$ asserted in $M$, combines these sets to form a set of equality sets that are complete for $\psi$, and generates an instantiation for each equality set produced in this manner.

\begin{minipage}[t]{.4\linewidth}
\begin{program}
\PROC |generate\_inst|(L_1, \ldots, L_n, M) \BODY
S_{L1} \ldots S_{Ln};
\DOFOR i=1 \TO n
  S_{Li} := calc\_e\_ind\_lit( L_i, M );
\ENDFOR
S := combine( S_{L1}, \ldots, S_{Ln} );
\RETURN \{ E^\varepsilon \mid E \in S, E \text{ is complete for $\psi$ } \};
\ENDPROC
\end{program}
\end{minipage}
\begin{minipage}[t]{.4\linewidth}
\begin{program}
\PROC |combine|(S_1, \ldots, S_n) \BODY
S := \emptyset;
\DOFOR i=1 \TO n
  S := S \cup S_i \cup merge( S, S_i )
\ENDFOR
\RETURN S;
\ENDPROC
\end{program}
\end{minipage}

\paragraph{Third Iteration: A Literal is Partially $E$-Induced}

If no instantiations are produced on the second iteration, we try to find an instantiation that is relevant in finding a match for $L[\bar{e}]$.
In this iteration, we do not require that $L[\bar{e}]$ is $E$-induced in $M$ for some $E$.
For example, say we have that $f( a, e )$ and $f( b, c )$ are in the same equivalence class and that $M \not\models a = b$.  
Here, we may wish to try the instantiation $c/e$, since doing partially matches our pattern to a relevant concrete ground term.

This iteration works in a similar fashion to the previous one, where instead of generating equality sets $E$ that induce $L_i$, we generate equality sets for which part of $L_i$ is matched to a concrete ground term.
First, let us define the method $partial\_calc\_e\_ind$:

\begin{minipage}[t]{.4\linewidth}
\begin{program}
\PROC |partial\_calc\_e\_ind|(t,g,M) \BODY
\ENDPROC
\end{program}
\end{minipage}


\paragraph{Fourth Iteration: Match all Triggers with all Ground Terms}

\end{document}

